%%%%%%%%%%%%%%%%%%%%%%%%%%%%%%%%%%%%%%%%%
% Wenneker Article
% LaTeX Template
% Version 2.0 (28/2/17)
%
% This template was downloaded from:
% http://www.LaTeXTemplates.com
%
% Authors:
% Vel (vel@LaTeXTemplates.com)
% Frits Wenneker
%
% License:
% CC BY-NC-SA 3.0 (http://creativecommons.org/licenses/by-nc-sa/3.0/)
%
%%%%%%%%%%%%%%%%%%%%%%%%%%%%%%%%%%%%%%%%%

%----------------------------------------------------------------------------------------
%	PACKAGES AND OTHER DOCUMENT CONFIGURATIONS
%----------------------------------------------------------------------------------------

\documentclass[10pt, a4paper, twocolumn]{article} % 10pt font size (11 and 12 also possible), A4 paper (letterpaper for US letter) and two column layout (remove for one column)

%%%%%%%%%%%%%%%%%%%%%%%%%%%%%%%%%%%%%%%%%
% Wenneker Article
% Structure Specification File
% Version 1.0 (28/2/17)
%
% This file originates from:
% http://www.LaTeXTemplates.com
%
% Authors:
% Frits Wenneker
% Vel (vel@LaTeXTemplates.com)
%
% License:
% CC BY-NC-SA 3.0 (http://creativecommons.org/licenses/by-nc-sa/3.0/)
%
%%%%%%%%%%%%%%%%%%%%%%%%%%%%%%%%%%%%%%%%%

%----------------------------------------------------------------------------------------
%	PACKAGES AND OTHER DOCUMENT CONFIGURATIONS
%----------------------------------------------------------------------------------------

\usepackage[english,main=russian]{babel} % English language hyphenation

\usepackage{microtype} % Better typography

\usepackage{amsmath,amsfonts,amsthm} % Math packages for equations

\usepackage[svgnames]{xcolor} % Enabling colors by their 'svgnames'

\usepackage[hang, small, labelfont=bf, up, textfont=it]{caption} % Custom captions under/above tables and figures

\usepackage{booktabs} % Horizontal rules in tables

\usepackage{lastpage} % Used to determine the number of pages in the document (for "Page X of Total")

\usepackage{graphicx} % Required for adding images

\usepackage{enumitem} % Required for customising lists
\setlist{noitemsep} % Remove spacing between bullet/numbered list elements

\usepackage{sectsty} % Enables custom section titles
\allsectionsfont{\usefont{OT1}{phv}{b}{n}} % Change the font of all section commands (Helvetica)

%----------------------------------------------------------------------------------------
%	MARGINS AND SPACING
%----------------------------------------------------------------------------------------

\usepackage{geometry} % Required for adjusting page dimensions

\geometry{
	top=1cm, % Top margin
	bottom=1.5cm, % Bottom margin
	left=2cm, % Left margin
	right=2cm, % Right margin
	includehead, % Include space for a header
	includefoot, % Include space for a footer
	%showframe, % Uncomment to show how the type block is set on the page
}

\setlength{\columnsep}{7mm} % Column separation width

%----------------------------------------------------------------------------------------
%	FONTS
%----------------------------------------------------------------------------------------

\usepackage[T1]{fontenc} % Output font encoding for international characters
\usepackage[utf8]{inputenc} % Required for inputting international characters

\usepackage{XCharter} % Use the XCharter font

%----------------------------------------------------------------------------------------
%	HEADERS AND FOOTERS
%----------------------------------------------------------------------------------------

\usepackage{fancyhdr} % Needed to define custom headers/footers
\pagestyle{fancy} % Enables the custom headers/footers

\renewcommand{\headrulewidth}{0.0pt} % No header rule
\renewcommand{\footrulewidth}{0.4pt} % Thin footer rule

\renewcommand{\sectionmark}[1]{\markboth{#1}{}} % Removes the section number from the header when \leftmark is used

%\nouppercase\leftmark % Add this to one of the lines below if you want a section title in the header/footer

% Headers
\lhead{} % Left header
\chead{\textit{\thetitle}} % Center header - currently printing the article title
\rhead{} % Right header

% Footers
\lfoot{} % Left footer
\cfoot{} % Center footer
\rfoot{\footnotesize Page \thepage\ of \pageref{LastPage}} % Right footer, "Page 1 of 2"

\fancypagestyle{firstpage}{ % Page style for the first page with the title
	\fancyhf{}
	\renewcommand{\footrulewidth}{0pt} % Suppress footer rule
}

%----------------------------------------------------------------------------------------
%	TITLE SECTION
%----------------------------------------------------------------------------------------

\newcommand{\authorstyle}[1]{{\large\usefont{OT1}{phv}{b}{n}\color{DarkRed}#1}} % Authors style (Helvetica)

\newcommand{\institution}[1]{{\footnotesize\usefont{OT1}{phv}{m}{sl}\color{Black}#1}} % Institutions style (Helvetica)

\usepackage{titling} % Allows custom title configuration

\newcommand{\HorRule}{\color{DarkGoldenrod}\rule{\linewidth}{1pt}} % Defines the gold horizontal rule around the title

\pretitle{
	\vspace{-30pt} % Move the entire title section up
	\HorRule\vspace{10pt} % Horizontal rule before the title
	\fontsize{32}{36}\usefont{OT1}{phv}{b}{n}\selectfont % Helvetica
	\color{DarkRed} % Text colour for the title and author(s)
}

\posttitle{\par\vskip 15pt} % Whitespace under the title

\preauthor{} % Anything that will appear before \author is printed

\postauthor{ % Anything that will appear after \author is printed
	\vspace{10pt} % Space before the rule
	\par\HorRule % Horizontal rule after the title
	\vspace{20pt} % Space after the title section
}

%----------------------------------------------------------------------------------------
%	ABSTRACT
%----------------------------------------------------------------------------------------

\usepackage{lettrine} % Package to accentuate the first letter of the text (lettrine)
\usepackage{fix-cm}	% Fixes the height of the lettrine

\newcommand{\initial}[1]{ % Defines the command and style for the lettrine
	\lettrine[lines=3,findent=4pt,nindent=0pt]{% Lettrine takes up 3 lines, the text to the right of it is indented 4pt and further indenting of lines 2+ is stopped
		\color{DarkGoldenrod}% Lettrine colour
		{#1}% The letter
	}{}%
}

\usepackage{xstring} % Required for string manipulation

\newcommand{\lettrineabstract}[1]{
	\StrLeft{#1}{1}[\firstletter] % Capture the first letter of the abstract for the lettrine
	\initial{\firstletter}\textbf{\StrGobbleLeft{#1}{1}} % Print the abstract with the first letter as a lettrine and the rest in bold
}

\def\mybibname{\fontencoding{T2A}\selectfont
\CYRL\cyri\cyrt\cyre\cyrr\cyra\cyrt\cyru\cyrr\cyra}
\def\myshortname{%%%
\fontencoding{T2A}\selectfont
  \CYRM\cyro\cyrd\cyre\cyrl.
  \cyri{}
  \cyra\cyrn\cyra\cyrl\cyri\cyrz{}
  \cyri\cyrn\cyrf\cyro\cyrr\cyrm.
  \cyrs\cyri\cyrs\cyrt\cyre\cyrm.{}
}
\def\mylongname{%%%
\fontencoding{T2A}\selectfont
  \CYRM\cyro\cyrd\cyre\cyrl\cyri\cyrr\cyro\cyrv\cyra\cyrn\cyri\cyre{}
  \cyri{}
  \cyra\cyrn\cyra\cyrl\cyri\cyrz{}
  \cyri\cyrn\cyrf\cyro\cyrr\cyrm\cyra\cyrc\cyri\cyro\cyrn\cyrn\cyrery\cyrh{}
  \cyrs\cyri\cyrs\cyrt\cyre\cyrm{}
}
\def\myrecname{\fontencoding{T2A}\selectfont
\cyrp\cyro\cyrl\cyru\cyrch\cyre\cyrn\cyra}
\def\myvolname{\fontencoding{T2A}\selectfont
\CYRT.}
\def\myUDCname{\fontencoding{T2A}\selectfont
\CYRU\CYRD\CYRK}
%----------------------------------------------------------------------------------------
%	BIBLIOGRAPHY
%----------------------------------------------------------------------------------------

%\usepackage[backend=bibtex,style=authoryear,natbib=true]{biblatex} % Use the bibtex backend with the authoryear citation style (which resembles APA)

%\addbibresource{example.bib} % The filename of the bibliography

%\usepackage[autostyle=true]{csquotes} % Required to generate language-dependent quotes in the bibliography
 % Specifies the document structure and loads requires packages

%----------------------------------------------------------------------------------------
%	ARTICLE INFORMATION
%----------------------------------------------------------------------------------------

\title{Implementing Cross-Section Performance Reversion} % The article title

\author{
	\authorstyle{Valery Ovchinnikov} % Authors
}

% Example of a one line author/institution relationship
%\author{\newauthor{John Marston} \newinstitution{Universidad Nacional Autónoma de México, Mexico City, Mexico}}

\date{\today} % Add a date here if you would like one to appear underneath the title block, use \today for the current date, leave empty for no date

%----------------------------------------------------------------------------------------

\begin{document}

\maketitle % Print the title

\thispagestyle{firstpage} % Apply the page style for the first page (no headers and footers)

%----------------------------------------------------------------------------------------
%	ABSTRACT
%----------------------------------------------------------------------------------------

%----------------------------------------------------------------------------------------
%	ARTICLE CONTENTS
%----------------------------------------------------------------------------------------

\section{Введение}
Mean-revertion широко используемая идея для стратегий в основе которой лежит идея о том, что доходности или цены должны
вернуться к своему нормальному поведению. Нормальным поведением может быть вовзрат к долгосрочному среднему цена,
но может быть и что-то посложнее. Например для fx spot инструментов известно, что  если вчера цена ушла вверх, то завтра
она скорее всего пойдет вниз и наоборот, то есть вернется к среднему. В данной статье для получения нормального
поведения доходностей акций используется кросс-корреляция между ними. Использование кросс-корреляции позволяет оценить
нормальное поведение доходностей одной акции при условии, что рынок ведёт себя так, как мы наблюдаем в данный момент.
%------------------------------------------------

\section{Алгоритм}
Идея алгоритма заключается в том, что логарифмы доходности акций должны возвращаться к модельным ожидаемым значениям.\\
Прежде всего фильтруется ряд доходностей. Для этого используется модель GARCH. Среднее моделируется константой.
Распределение доходностей считаем нормальным. Это необходимо, чтобы в дальнейшем воспользоваться явной простой формулой для
условного матожидания.
$$r_i(t) = \mu_i +  \epsilon_i(t)$$\\
$$\epsilon(t) \~ \sigma(t) \cdot N(0, R)$$\\
$$\sigma_i^2(t) = \omega_i + \alpha_i \cdot \epsilon_i^2(t-1) + \beta_i \cdot \sigma_i^2(t-1)$$\\
Подбор параметров AR и MV для модели GARCH делается простым перебором, в результате которого выбираются параметры,
с которыми модель выдает наименьший показатель AIC. Такой перебор делается над всеми рядами данных и выбирается наиболее
часто встречающийся набор.\\
Для фильтрованных рядов рассчитывается корреляционная матрица. Она понадобится для расчета условного матожидания доходностей акций.
$$\oveline{\mu} = \mu_i + \sigma_i(t) \cdot R_{i,J} \cdot R_J^{-1} \cdot \freq{r_{i,J} - \mu_J}{\sigma_{i,J}}$$,
где $J = {1,\.\.\.,n} \\ {i}$, $R_J$ это матрица без $i$-ой строки и $i$-го столбца, $R_{i,J}$ это $i$-я строка матрицы
$R$, без $i$-го столбца.
Чтобы уменьшить корреляции между рядами доходностей акций, из каждого ряда вычитается beta с рынком:
$r'_i(t) = r_i(t) - \beta_{i, mkt} * r_{mkt}(t)$
Таким образом вычищаются совместные движения доходностей отдельных акций с рынком.\\
Сигнал (aka альфа) получается следующим: $\alpha_i(t) = \overline{\mu} + (\overline{\mu} - r_i(t))$. То есть на
следующем шаге ожидается доходность $\overline{\mu}$ и дополнительная доходность от возвращения к среднему с прошлого
шага $\overline{\mu} - r_i$.\\
Модель GARCH перефичивается каждые 30 дней. Размер окна выбран произвольный, не оптимизирован никак, 200 дней,
чтобы зацепить почти весь год.

\section{Возможные улучшения}
В статье описан процесс хеджирования факторов, таких как momentum, value, volatility factor. Так же предлагается
спользовать самофинансируемую стратегию, т.е. в каждый момент времени мы ничего не занимаем и не откладываем. И минимизировать
exposure к рынку, т.е. выбрать такие веса, чтобы взвешенная сумма коэффициентов beta была близка к нулю. Общеизвестный
факт, что хеджирование не увеличивает матожидание стратегии, а может только уменьшить волатильность. Так как в данном кейсе
не стояла цель получить высокий коэффициент Шарпа, и времени было мало, было решено не реализовывать эту часть статьи.\\
Другим возможным улучшением является изменения алгоритма рсчета обратной матрицы. По словам авторов использование псевдо
обратной матрицы, полученной с помощью Woodbery identity, позволяет уменьшить численные ошибки в алгоритме, а также ускорить
вычисление сигнала в несколько раз за счет инвертирования матриц маленьких рангов (2) вместо ковариационной матрицы
(ранг равен количеству акций в портфеле).\\
В текущей реализации достаточно поменять один параметр в коде, чтобы для моделирования среднего использовалась более сложная
ARMA модел вместо константы. Исследование рядов, отфильтрованных таким способом показало, что автокорреляция в них ниже,
чем при фильтрации константной моделью, однако в статье использована именно константа.
%---------------------------------------------------
\end{document}

