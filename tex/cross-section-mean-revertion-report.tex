%%%%%%%%%%%%%%%%%%%%%%%%%%%%%%%%%%%%%%%%%
% Wenneker Article
% LaTeX Template
% Version 2.0 (28/2/17)
%
% This template was downloaded from:
% http://www.LaTeXTemplates.com
%
% Authors:
% Vel (vel@LaTeXTemplates.com)
% Frits Wenneker
%
% License:
% CC BY-NC-SA 3.0 (http://creativecommons.org/licenses/by-nc-sa/3.0/)
%
%%%%%%%%%%%%%%%%%%%%%%%%%%%%%%%%%%%%%%%%%

%----------------------------------------------------------------------------------------
%	PACKAGES AND OTHER DOCUMENT CONFIGURATIONS
%----------------------------------------------------------------------------------------

\documentclass[10pt, a4paper, twocolumn]{article} % 10pt font size (11 and 12 also possible), A4 paper (letterpaper for US letter) and two column layout (remove for one column)

\input{structure.tex} % Specifies the document structure and loads requires packages

%----------------------------------------------------------------------------------------
%	ARTICLE INFORMATION
%----------------------------------------------------------------------------------------

\title{Implementing Cross-Section Performance Reversion} % The article title

\author{
	\authorstyle{Valery Ovchinnikov} % Authors
}

% Example of a one line author/institution relationship
%\author{\newauthor{John Marston} \newinstitution{Universidad Nacional Autónoma de México, Mexico City, Mexico}}

\date{\today} % Add a date here if you would like one to appear underneath the title block, use \today for the current date, leave empty for no date

%----------------------------------------------------------------------------------------

\begin{document}

\maketitle % Print the title

\thispagestyle{firstpage} % Apply the page style for the first page (no headers and footers)

%----------------------------------------------------------------------------------------
%	ABSTRACT
%----------------------------------------------------------------------------------------

%----------------------------------------------------------------------------------------
%	ARTICLE CONTENTS
%----------------------------------------------------------------------------------------

\section{Введение}
Mean-revertion широко используемая идея для стратегий в основе которой лежит идея о том, что доходности или цены должны
вернуться к своему нормальному поведению. Нормальным поведением может быть вовзрат к долгосрочному среднему цена,
но может быть и что-то посложнее. Например для fx spot инструментов известно, что  если вчера цена ушла вверх, то завтра
она скорее всего пойдет вниз и наоборот, то есть вернется к среднему. В данной статье для получения нормального
поведения доходностей акций используется кросс-корреляция между ними. Использование кросс-корреляции позволяет оценить
нормальное поведение доходностей одной акции при условии, что рынок ведёт себя так, как мы наблюдаем в данный момент.
%------------------------------------------------

\section{Алгоритм}
Идея алгоритма заключается в том, что логарифмы доходности акций должны возвращаться к модельным ожидаемым значениям.\\
Прежде всего фильтруется ряд доходностей. Для этого используется модель GARCH. Среднее моделируется константой.
Распределение доходностей считаем нормальным. Это необходимо, чтобы в дальнейшем воспользоваться явной простой формулой для
условного матожидания.
$$r_i(t) = \mu_i +  \epsilon_i(t)$$\\
$$\epsilon(t) \~ \sigma(t) \cdot N(0, R)$$\\
$$\sigma_i^2(t) = \omega_i + \alpha_i \cdot \epsilon_i^2(t-1) + \beta_i \cdot \sigma_i^2(t-1)$$\\
Подбор параметров AR и MV для модели GARCH делается простым перебором, в результате которого выбираются параметры,
с которыми модель выдает наименьший показатель AIC. Такой перебор делается над всеми рядами данных и выбирается наиболее
часто встречающийся набор.\\
Для фильтрованных рядов рассчитывается корреляционная матрица. Она понадобится для расчета условного матожидания доходностей акций.
$$\oveline{\mu} = \mu_i + \sigma_i(t) \cdot R_{i,J} \cdot R_J^{-1} \cdot \freq{r_{i,J} - \mu_J}{\sigma_{i,J}}$$,
где $J = {1,\.\.\.,n} \\ {i}$, $R_J$ это матрица без $i$-ой строки и $i$-го столбца, $R_{i,J}$ это $i$-я строка матрицы
$R$, без $i$-го столбца.
Чтобы уменьшить корреляции между рядами доходностей акций, из каждого ряда вычитается beta с рынком:
$r'_i(t) = r_i(t) - \beta_{i, mkt} * r_{mkt}(t)$
Таким образом вычищаются совместные движения доходностей отдельных акций с рынком.\\
Сигнал (aka альфа) получается следующим: $\alpha_i(t) = \overline{\mu} + (\overline{\mu} - r_i(t))$. То есть на
следующем шаге ожидается доходность $\overline{\mu}$ и дополнительная доходность от возвращения к среднему с прошлого
шага $\overline{\mu} - r_i$.\\
Модель GARCH перефичивается каждые 30 дней. Размер окна выбран произвольный, не оптимизирован никак, 200 дней,
чтобы зацепить почти весь год.

\section{Возможные улучшения}
В статье описан процесс хеджирования факторов, таких как momentum, value, volatility factor. Так же предлагается
спользовать самофинансируемую стратегию, т.е. в каждый момент времени мы ничего не занимаем и не откладываем. И минимизировать
exposure к рынку, т.е. выбрать такие веса, чтобы взвешенная сумма коэффициентов beta была близка к нулю. Общеизвестный
факт, что хеджирование не увеличивает матожидание стратегии, а может только уменьшить волатильность. Так как в данном кейсе
не стояла цель получить высокий коэффициент Шарпа, и времени было мало, было решено не реализовывать эту часть статьи.\\
Другим возможным улучшением является изменения алгоритма рсчета обратной матрицы. По словам авторов использование псевдо
обратной матрицы, полученной с помощью Woodbery identity, позволяет уменьшить численные ошибки в алгоритме, а также ускорить
вычисление сигнала в несколько раз за счет инвертирования матриц маленьких рангов (2) вместо ковариационной матрицы
(ранг равен количеству акций в портфеле).\\
В текущей реализации достаточно поменять один параметр в коде, чтобы для моделирования среднего использовалась более сложная
ARMA модел вместо константы. Исследование рядов, отфильтрованных таким способом показало, что автокорреляция в них ниже,
чем при фильтрации константной моделью, однако в статье использована именно константа.
%---------------------------------------------------
\end{document}

